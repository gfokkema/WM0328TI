\documentclass[a4paper]{article}

\begin{document}

\section*{Scenarios}
\subsection*{Scenario 5.2: Obtaining pirated software abroad}
Carol buys a duplicated copy of software while she's abroad.
Should customs find said copy when returning to USA,
is Carol then punishable? Should customs be allowed to confiscate her copy?

\subsection*{Scenario 5.2: Free software that follows proprietary software}
The company Bingo has invested a substantial amount of money in developing
some software product. Now another company, Pete's Software,
studies Bingo's solution and releases similar software under GPL.
Pete's Software aims to be profitable by selling support and services.
Should this be allowed?

\subsection*{Scenario 5.3: Using public domain software in proprietary software}
Earl Eniac has released software under a CC licencse.
Now Jake Jasper revises this software, also sending Earl his revision.
He also copyrights his revision and starts selling it as proprietary.
Is Jake obligated to offer his revision under CC?

\section*{Introduction}
\begin{itemize}
\item Software protected by copyright, trade secrecy or patent law becomes
      \textit{(Digital) Intellectual Property} or IP for short.
\item Conventional laws for IP (books, records) don't really apply to
      Digital IP (software, music) mainly because of ease of reproducibility.
\item Ownership and control of Digital IP differs greatly from IP,
      there is no concept of stealing.
\end{itemize}

IP laws distinguish between the following aspects of software:
algorithms, object (binary) code and source code.

\section*{Setting the stage}
A solution to scenario 5.2 could be to just give Bingo a legal right to
prevent a company like Pete's Software from creating software like it.
This turns out to be hard, while there are also legitimate reasons not to do so.

Heavily contested, lots of lawsuits, blablabla.

\section*{Protecting property rights}
Description of copyright, trade secrecy and patent,
together proprietary software (PS). Afterwards FOSS.

\subsection*{Copyright}
Copyright law is a form of ownership that prohibits others from reproducing
copyrighted work, distributing copies of it and displaying or performing
the copyrighted work publicly, without first obtaining permission.
Copyright holds for as long as the author lives plus 70 years (before 1998 50 years).
US copyright comes from the US Constitution which says:
"To promote progress of science and arts, authors and inventors may,
for limited time, hold exclusive rights over their writings and discoveries."
Exclusive rights also means rights can be sold for a fee.

Copyright law further specifies this by making a distinction between an idea
and the expression of an idea. Only the expression of an idea is copyrightable.
\begin{itemize}
\item Algorithms (the most valuable part of software) are generally
      thought of as an idea and not an expression and are not copyrightable.
\item Algorithms expressed as source code or object code become an
      expression of an idea (literary work) and therefore are copyrightable.
\end{itemize}
The task of creating new source and object code may be negligible once
the general approach has already been outlined in proprietary software,
as was illustrated in scenario 5.2. \\
The distinction between idea and expression doesn't capture software
functionality or behaviour, which allows competitors to slightly alter
the `expression' of the idea while offering similar functionality and behaviour. \\
Another issue is how much similarity there should be between the source and object
code of a program for it to be infringement. There are many ways to alter source
code while maintaining functionality of the object code (ie. change varnames).
The copyright holder must prove infringement, which is hard because of this. \\
Furthermore, if someone wrote identical software without knowledge of
copyright holder's work, there is no infringement. \\


\end{document}
